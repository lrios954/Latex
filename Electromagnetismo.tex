
\documentclass[paper=a4, fontsize=11pt]{scrartcl} % A4 paper and 11pt font size

\usepackage[T1]{fontenc} % Use 8-bit encoding that has 256 glyphs
\usepackage{fourier} % Use the Adobe Utopia font for the document - comment this line to return to the LaTeX default
\usepackage[english]{babel} % English language/hyphenation
\usepackage{amsmath,amsfonts,amsthm} % Math packages

\usepackage{lipsum} % Used for inserting dummy 'Lorem ipsum' text into the template

\usepackage{sectsty} % Allows customizing section commands
\allsectionsfont{\centering \normalfont\scshape} % Make all sections centered, the default font and small caps

\usepackage{fancyhdr} % Custom headers and footers
\pagestyle{fancyplain} % Makes all pages in the document conform to the custom headers and footers
\fancyhead{} % No page header - if you want one, create it in the same way as the footers below
\fancyfoot[L]{} % Empty left footer
\fancyfoot[C]{} % Empty center footer
\fancyfoot[R]{\thepage} % Page numbering for right footer
\renewcommand{\headrulewidth}{0pt} % Remove header underlines
\renewcommand{\footrulewidth}{0pt} % Remove footer underlines
\setlength{\headheight}{13.6pt} % Customize the height of the header

\numberwithin{equation}{section} % Number equations within sections (i.e. 1.1, 1.2, 2.1, 2.2 instead of 1, 2, 3, 4)
\numberwithin{figure}{section} % Number figures within sections (i.e. 1.1, 1.2, 2.1, 2.2 instead of 1, 2, 3, 4)
\numberwithin{table}{section} % Number tables within sections (i.e. 1.1, 1.2, 2.1, 2.2 instead of 1, 2, 3, 4)

\setlength\parindent{0pt} % Removes all indentation from paragraphs - comment this line for an assignment with lots of text

%----------------------------------------------------------------------------------------
%	TITLE SECTION
%----------------------------------------------------------------------------------------

\newcommand{\horrule}[1]{\rule{\linewidth}{#1}} % Create horizontal rule command with 1 argument of height

\title{	
\normalfont \normalsize 
\textsc{Universidad de Los Andes} \\ [25pt] % Your university, school and/or department name(s)
\horrule{0.5pt} \\[0.4cm] % Thin top horizontal rule
\huge Spherical Electromagnetic Waves  \\ % The assignment title
\horrule{2pt} \\[0.5cm] % Thick bottom horizontal rule
}

\author{Leonardo Rios,  Nicolas Hayek,  Catalina} % Your name

\date{\normalsize{Nov 01, 2013}} % Today's date or a custom date



\begin{document}

\maketitle % Print the title



\section*{SPHERICAL WAVES}

The most easily treated solutions for the Maxwell's equations are those in one dimension or those written as plane wave equations, however mathematically one can treat an electromagnetic wave in any system of coordinates. In all the coordinate systems available, the spherical coordinates are very useful for representing some phenomena in physics. 

\paragraph*{Wave Equation:}
We consider the same wave equation we know but now, in spherical coordinates:

\begin{equation}
\nabla^2 E-\frac{1}{c^2}\frac{\partial^2 E}{\partial t^2} =0 
\end{equation}
For monochromatic waves, the operator $\frac{\partial^2 }{\partial t^2}$ becomes $-w^2$: \\
\begin{equation}
\nabla^2 E(r)+\frac{w^2}{c^2}E(r) =0 \label{eq2}
\end{equation} 
Using these vector identities:

\begin{equation}
\nabla^2 E=-\nabla \times \nabla \times E + \nabla\nabla \cdot E 
\end{equation} 
\begin{equation}
\nabla \times (F \times G)=F\nabla\cdot G -G \nabla \cdot F + (G\cdot\nabla)F - (F\cdot\nabla)G
\end{equation} 
\begin{equation}
\nabla(F\cdot G)=(F \cdot \nabla)G + (G \cdot \nabla)F+ F\times \nabla \times G + G \times \nabla \times F
\end{equation}

Assuming $\psi$ as a solution for (\ref{eq2}) and $E=r\times \nabla \psi$ we obtain a variation of the Helmholtz equation:
\begin{equation}
\nabla \times \nabla \times (r\times \nabla \psi)= \frac{w^2}{c^2} r\times \nabla \psi
\end{equation}

Now that we know $r\times \nabla \psi$ is a solution for the wave equation we only need to find out the specifics solutions that are able to represent electromagnetic waves. The electric field must satisfy Maxwell's equations, it means:

\begin{equation}
\nabla \times E=iwB
\end{equation}

\begin{equation}
B=-i\frac{1}{w} \nabla \times E
\end{equation}

For solving these equations in the spherical coordinates we need to use the method of separation of variables. Let's take (\ref{eq2}) using $\nabla^2$ in the respective coordinates


\begin{equation}
\nabla ^2\phi = \left( {\underbrace {\frac{\partial ^2\phi }{\partial r^2} +
\frac{2}{r}\frac{\partial \phi }{\partial
r}}_{\frac{1}{^{r^2}}\frac{\partial }{\partial r}\left( {r^2\frac{\partial
\phi }{\partial r}} \right)} + \frac{1}{r^2\sin \theta }\frac{\partial
}{\partial \theta }\left( {\sin \theta \frac{\partial \phi }{\partial \theta
}} \right) + \frac{1}{r^2\sin ^2\theta }\frac{\partial ^2\phi }{\partial
\varphi ^2}} \right)
\end{equation}
where the solution for $\psi$ is assumed to have he form:
\begin{equation}
\psi=R(r)\Theta(\theta) \Phi(\phi)
\end{equation}

This, almost immediately, takes to the solutions for each function $R(r)$ , $\Theta(\theta)$ , $\Phi(\phi)$ where the Legendre Polynomials, $P_{\ell}^{m}\left(\cos\theta\right)$, are solutions for the radial independent part and the Spherical Bessel and Neumann Functions are solutions to the radial differential equation.


\paragraph*{Parrafo:}



\begin{thebibliography}{2}
\bibitem{lib} Reitz, Milford, Christy: Foundations of electromagnetic theory. Addison-Wesley, 1993

\end{thebibliography}
\end{document}